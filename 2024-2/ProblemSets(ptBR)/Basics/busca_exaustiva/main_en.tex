\documentclass{article}
\usepackage{amsmath}
\usepackage{amssymb}
\usepackage{algorithm}
\usepackage{algorithmic}
\usepackage{url}
\usepackage[margin=0.8in]{geometry}

\title{Problem Set: BFS and DFS}
\author{Prof. Rodrigo Silva}
\date{}

\begin{document}

\maketitle

\section*{Introduction}

This problem set breaks down the implementation of BFS (Breadth-First Search) and DFS (Depth-First Search) into smaller parts. The goal is to help students gradually implement each component and then combine them to form the full algorithms.

\section*{Part 1: Representing the Graph}

\subsection*{Problem 1.1: Graph Representation}

Write a function to represent a graph using a dictionary, where each key is a node and the value is a list of its neighbors. (See \url{https://www.w3schools.com/python/python_dictionaries.asp} to learn how to use dictionaries (the data structure used to represent the graph) in Python).

\begin{verbatim}
Graph = {
    'A': ['B', 'C'],
    'B': ['A', 'D'],
    'C': ['A', 'E'],
    'D': ['B'],
    'E': ['C']
}
\end{verbatim}

The function \texttt{add\_edge(node1, node2)} should update this dictionary representation.

\begin{algorithm}[H]
\caption{add\_edge(node1, node2)}
\begin{algorithmic}
    \IF {node1 is not in Graph}
        \STATE Graph[node1] $\gets$ empty list
    \ENDIF
    \IF {node2 is not in Graph}
        \STATE Graph[node2] $\gets$ empty list
    \ENDIF
    \STATE append node2 to Graph[node1]
    \STATE append node1 to Graph[node2] \COMMENT{since the graph is undirected}
\end{algorithmic}
\end{algorithm}

---

\section*{Part 2: BFS - Breadth-First Search}

\subsection*{Problem 2.1: Initializing the Queue}

Write a function that initializes a queue for BFS using a Python list.

\begin{algorithm}[H]
\caption{Initialize\_Queue(start\_node)}
\begin{algorithmic}
    \STATE queue $\gets$ [start\_node] \COMMENT{Initialize queue with start\_node}
    \RETURN queue
\end{algorithmic}
\end{algorithm}

---

\subsection*{Problem 2.2: Handling Neighbors}

Write a function to visit a node and enqueue all its unvisited neighbors.

\begin{algorithm}[H]
\caption{Visit\_Neighbors(graph, node, visited, queue)}
\begin{algorithmic}
    \FOR{each neighbor in graph[node]}
        \IF{neighbor is not in visited}
            \STATE queue.append(neighbor) \COMMENT{enqueue neighbor}
            \STATE visited.add(neighbor)
        \ENDIF
    \ENDFOR
\end{algorithmic}
\end{algorithm}

---

\subsection*{Problem 2.3: BFS Complete}

Now, put everything together to complete the BFS algorithm.

\begin{algorithm}[H]
\caption{BFS(graph, start\_node)}
\begin{algorithmic}
    \STATE visited $\gets$ set() \COMMENT{empty set}
    \STATE queue $\gets$ Initialize\_Queue(start\_node)
    \STATE visited.add(start\_node)
    
    \WHILE{queue is not empty}
        \STATE node $\gets$ queue.pop(0) \COMMENT{dequeue from front}
        \STATE print "Visited", node
        \STATE Visit\_Neighbors(graph, node, visited, queue)
    \ENDWHILE
\end{algorithmic}
\end{algorithm}

---

\section*{Part 3: DFS - Depth-First Search}

\subsection*{Problem 3.1: Initializing the Stack}

Write a function that initializes a stack for DFS using a Python list.

\begin{algorithm}[H]
\caption{Initialize\_Stack(start\_node)}
\begin{algorithmic}
    \STATE stack $\gets$ [start\_node] \COMMENT{Initialize stack with start\_node}
    \RETURN stack
\end{algorithmic}
\end{algorithm}

---

\subsection*{Problem 3.2: Handling Neighbors in DFS}

Write a function to visit a node and push all its unvisited neighbors to the stack.

\begin{algorithm}[H]
\caption{Visit\_Neighbors\_DFS(graph, node, visited, stack)}
\begin{algorithmic}
    \FOR{each neighbor in graph[node]}
        \IF{neighbor is not in visited}
            \STATE stack.append(neighbor) \COMMENT{push neighbor onto stack}
            \STATE visited.add(neighbor)
        \ENDIF
    \ENDFOR
\end{algorithmic}
\end{algorithm}

---

\subsection*{Problem 3.3: DFS Complete}

Now, put everything together to complete the DFS algorithm.

\begin{algorithm}[H]
\caption{DFS(graph, start\_node)}
\begin{algorithmic}
    \STATE visited $\gets$ set() \COMMENT{empty set}
    \STATE stack $\gets$ Initialize\_Stack(start\_node)
    \STATE visited.add(start\_node)

    \WHILE{stack is not empty}
        \STATE node $\gets$ stack.pop() \COMMENT{pop from the stack}
        \STATE print "Visited", node
        \STATE Visit\_Neighbors\_DFS(graph, node, visited, stack)
    \ENDWHILE
\end{algorithmic}
\end{algorithm}

---

\section*{Part 4: Recursive DFS (Optional)}

\subsection*{Problem 4.1: Recursive DFS}

Write a recursive function to perform DFS.

\begin{algorithm}[H]
\caption{DFS\_Recursive(graph, node, visited)}
\begin{algorithmic}
    \STATE visited.add(node)
    \STATE print "Visited", node

    \FOR{each neighbor in graph[node]}
        \IF{neighbor is not in visited}
            \STATE DFS\_Recursive(graph, neighbor, visited)
        \ENDIF
    \ENDFOR
\end{algorithmic}
\end{algorithm}

---

\section*{Final Step: Putting It All Together}

At the end, students should test BFS and DFS on a sample graph by combining all the steps. This will help reinforce how each part fits into the whole algorithm.

\end{document}
