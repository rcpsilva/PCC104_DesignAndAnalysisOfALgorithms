\documentclass{article}
\usepackage[utf8]{inputenc}
\usepackage[brazil]{babel}
\usepackage[a4paper, margin=2.5cm]{geometry} 

\title{Exercícios Intermediários de Arrays e Matrizes}
\author{}
\date{}

\begin{document}

\maketitle

\begin{enumerate}
    \item Escreva um programa que receba um array de números inteiros e determine se o array está em ordem crescente, decrescente ou não está ordenado. Exiba uma mensagem indicando o resultado.
    
    \item Desenvolva um programa que receba uma matriz de números inteiros e retorne a transposta dessa matriz. A matriz transposta é obtida trocando-se as linhas por colunas.
    
    \item Crie um programa que receba dois arrays de números inteiros de mesmo tamanho e verifique se os dois arrays são anagramas um do outro (possuem os mesmos elementos, mas não necessariamente na mesma ordem). Exiba uma mensagem indicando o resultado.
    
    \item Crie um programa que receba uma matriz de números inteiros e calcule o determinante da matriz. Lembre-se de que o cálculo do determinante só faz sentido para matrizes quadradas.
    
    \item Escreva um programa que receba um array de números inteiros e retorne os três maiores elementos presentes no array, sem utilizar nenhuma função de ordenação pronta. Exiba esses três valores.
    
    \item Desenvolva um programa que receba uma matriz de números inteiros e identifique se ela é uma matriz simétrica. Uma matriz é simétrica se for quadrada e se for igual à sua transposta. Exiba uma mensagem indicando o resultado.
    
\end{enumerate}

\end{document}
