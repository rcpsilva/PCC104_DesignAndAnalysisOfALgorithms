\documentclass{article}
\usepackage[utf8]{inputenc}
\usepackage[margin=1.2in]{geometry}
\usepackage{hyperref}

\usepackage{listings}
\usepackage{xcolor}

\definecolor{codegreen}{rgb}{0,0.6,0}
\definecolor{codegray}{rgb}{0.5,0.5,0.5}
\definecolor{codepurple}{rgb}{0.58,0,0.82}
\definecolor{backcolour}{rgb}{0.95,0.95,0.92}

\lstdefinestyle{mystyle}{
    backgroundcolor=\color{backcolour},   
    commentstyle=\color{codegreen},
    keywordstyle=\color{magenta},
    numberstyle=\tiny\color{codegray},
    stringstyle=\color{codepurple},
    basicstyle=\ttfamily\footnotesize,
    breakatwhitespace=false,         
    breaklines=true,                 
    captionpos=b,                    
    keepspaces=true,                 
    numbers=left,                    
    numbersep=5pt,                  
    showspaces=false,                
    showstringspaces=false,
    showtabs=false,                  
    tabsize=2
}

\lstset{style=mystyle}


\usepackage{tikz}
\usetikzlibrary{positioning}

\usepackage{natbib}
\usepackage{graphicx}
\usepackage{amsmath}

\title{\vspace{-2 cm}Universidade Federal de Ouro Preto \\ PCC104 - Projeto e Análise de Algoritmos \\ Diminuir e Conquistar}
\author{Prof. Rodrigo Silva}
%\date{}


\begin{document}

\maketitle

\section*{Instruções}

\begin{itemize}
    \item Para cada conjunto de algoritmos o aluno deve escolher um.
    \item O aluno deve criar um repositório público no github com todos os códigos desenvolvidos.
    \item Em cada implementação o aluno deve:
    \begin{itemize}
        \item Apresentar 3 casos de teste
        \item Estar preparado para desenvolver a análise de custo 
        \item Estar preparado para responder perguntas sobre o seu próprio código, sobre o algoritmo e sobre o problema que o algoritmo resolve 
    \end{itemize} 
    \item As entrevistas devem ocorrer entre 
\end{itemize}

\section{Conjunto 1 - Dividir e Conquistar}
Capítulo 5 - \textit{Introduction to the Design and Analysis of Algorithms (3rd Edition)} by Anany Levitin

\begin{enumerate}
    \item Implemente o algoritmo MergeSort.
    \item Implemente o algoritmo QuickSort.
    \item Implemete um árvore binária e seus algoritmo de caminhamento:
    \begin{enumerate}
        \item pre-order
        \item pos-order
        \item in-order 
    \end{enumerate}
\end{enumerate}

\section{Conjunto 2 - Programação Dinâmica}
Capítulo 8 - \textit{Introduction to the Design and Analysis of Algorithms (3rd Edition)} by Anany Levitin

\begin{enumerate}
    \item Implemente os dois algoritmos baseados em programação dinâmica para o problem da mochila. (Seção 8.2)
\end{enumerate}

\section{Conjunto 3 - Algoritmos Gulosos}
Chapter 9 - \textit{Introduction to the Design and Analysis of Algorithms (3rd Edition)} by Anany Levitin

\begin{enumerate}
    \item Implemente o algoritmo de Prim.
    \item Implemente o algorimo de Kruskall.
    \item Implemente o algoritmo de Dijkstra.
\end{enumerate}

\section{Conjunto 4 - Backtracking}
Seção 12.1 - \textit{Introduction to the Design and Analysis of Algorithms (3rd Edition)} by Anany Levitin

\begin{enumerate}
    \item Implemente um algoritmo baseado em backtracking para o problema das n-rainhas.
    \item Implemente um algoritmo baseado em backtracking para o problem \textit(subset-sum).
\end{enumerate}

\section{Conjunto 5 - Branch and Bound}
Seção 12.2 - \textit{Introduction to the Design and Analysis of Algorithms (3rd Edition)} by Anany Levitin

\begin{enumerate}
    \item Implemente um algoritmo baseado em branch and bound para o problema da mochila.
    \item Implemente um algoritmo baseado em branch and bound para o problema do caixeiro viajante.  
\end{enumerate}

%\footnotetext{Livro - \textit{Introduction to the Design and Analysis of Algorithms (3rd Edition)}}

%\bibliographystyle{plain}
%\bibliography{references}
\end{document}

