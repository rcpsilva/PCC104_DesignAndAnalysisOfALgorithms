\documentclass[a4paper, 12pt]{article}
\usepackage[utf8]{inputenc}
\usepackage[portuguese]{babel}
\usepackage{amsmath}
\usepackage{geometry}

\geometry{a4paper, margin=1in}

\title{Lista de Exercícios de Programação}
\author{}
\date{}

\begin{document}

\maketitle

\vspace{-2cm}

\section*{Estruturas de Seleção (if...else) e (if...else) aninhados}

\begin{enumerate}
    \item Escreva um programa que receba um número inteiro como entrada e informe se ele é positivo, negativo ou zero.
    \item Crie um programa que receba três números e determine qual é o maior entre eles.
    \item Escreva um programa que receba a idade de uma pessoa e indique se ela pode votar (idade igual ou superior a 18 anos) ou se ainda não pode (idade inferior a 18 anos). Considere também a idade superior a 70 anos, onde o voto é opcional.
\end{enumerate}

\section*{Estruturas de Repetição (for) e (while)}

\begin{enumerate}
    \item Escreva um programa que exiba todos os números de 1 a 100.
    \item Crie um programa que calcule a soma dos números de 1 a 50 utilizando um laço de repetição.
    \item Escreva um programa que peça ao usuário para digitar um número positivo e continue solicitando até que o número digitado seja positivo.
\end{enumerate}

\section*{Manipulação de Arrays}

\begin{enumerate}
    \item Escreva um programa que receba 5 números inteiros e armazene-os em um array. Em seguida, exiba os números em ordem inversa.
    \item Crie um programa que leia um array de 10 elementos inteiros e informe a soma de todos os elementos.
    \item Escreva um programa que leia dois arrays de 5 elementos inteiros cada e crie um terceiro array que contenha os elementos do primeiro array seguidos pelos elementos do segundo array.
\end{enumerate}

\section*{Manipulação de Matrizes}

\begin{enumerate}
    \item Escreva um programa que dada uma matriz 3x3 e exiba a soma dos elementos da diagonal principal.
    \item Crie um programa que dada uma matriz 5x5 e multiplique todos os seus elementos por um número fornecido pelo usuário.
    \item Escreva um programa que dada duas matrizes 2x2 e exiba a matriz resultante da soma das duas matrizes.
\end{enumerate}

\section*{Recursão}

\begin{enumerate}
    \item Escreva uma função recursiva que receba um número inteiro positivo \( n \) e retorne o fatorial de \( n \) (denotado como \( n! \)).
    \item Crie uma função recursiva que receba um número inteiro \( n \) e exiba a sequência de Fibonacci até o \( n \)-ésimo termo.
    \item Escreva uma função recursiva que calcule a soma dos elementos de um array de números inteiros.
\end{enumerate}


\end{document}
