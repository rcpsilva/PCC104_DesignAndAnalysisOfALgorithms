\documentclass[12pt]{article}
\usepackage[margin=1in]{geometry} % Ajuste a margem alterando o valor
\usepackage{amsmath}
\usepackage{amssymb}
\usepackage[hyphens]{url}
\usepackage{hyperref}
\hypersetup{breaklinks=true}

\title{Recursão}
\author{PCC104 - Prof. Rodrigo Silva}
\date{}

\begin{document}

\maketitle
\section*{Leitura Recomendada}

\begin{itemize}
    \item Recursão \url{https://panda.ime.usp.br/pensepy/static/pensepy/12-Recursao/recursionsimple-ptbr.html}
\end{itemize}

\section*{Conjunto de Problemas}

\begin{enumerate}

    \item \textbf{Imprima de 1 a n sem usar laços.} \\
    Escreva uma função recursiva para imprimir os números de 1 a n sem usar laços. \\
    \textbf{Exemplo:} \\
    Entrada: n = 5 \\
    Saída: 1 2 3 4 5 \\
    \textbf{Caso de Teste:} \\
    Entrada: n = 7 \\
    Saída Esperada: 1 2 3 4 5 6 7
    
    \item \textbf{Imprima de N a 1 sem laço.} \\
    Escreva uma função recursiva para imprimir os números de N a 1 sem usar laços. \\
    \textbf{Exemplo:} \\
    Entrada: N = 5 \\
    Saída: 5 4 3 2 1 \\
    \textbf{Caso de Teste:} \\
    Entrada: N = 8 \\
    Saída Esperada: 8 7 6 5 4 3 2 1
    
    \item \textbf{Média de um vetor usando Recursão.} \\
    Crie uma função recursiva que calcula a média de um vetor de números. \\
    \textbf{Exemplo:} \\
    Entrada: arr = [1, 2, 3, 4, 5] \\
    Saída: 3 \\
    \textbf{Caso de Teste:} \\
    Entrada: arr = [10, 20, 30, 40, 50] \\
    Saída Esperada: 30
    
    \item \textbf{Soma de números naturais usando recursão.} \\
    Escreva uma função recursiva para calcular a soma dos primeiros n números naturais. \\
    \textbf{Exemplo:} \\
    Entrada: n = 5 \\
    Saída: 15 \\
    \textbf{Caso de Teste:} \\
    Entrada: n = 10 \\
    Saída Esperada: 55
    
    \item \textbf{Decimal para binário usando recursão.} \\
    Implemente uma função recursiva para converter um número decimal para seu equivalente binário. \\
    \textbf{Exemplo:} \\
    Entrada: n = 10 \\
    Saída: 1010 \\
    \textbf{Caso de Teste:} \\
    Entrada: n = 15 \\
    Saída Esperada: 1111
    
    \item \textbf{Soma dos elementos de um vetor usando recursão.} \\
    Escreva uma função recursiva para calcular a soma de todos os elementos de um vetor. \\
    \textbf{Exemplo:} \\
    Entrada: arr = [1, 2, 3, 4, 5] \\
    Saída: 15 \\
    \textbf{Caso de Teste:} \\
    Entrada: arr = [10, 20, 30, 40] \\
    Saída Esperada: 100
    
    \item \textbf{Imprima a string ao contrário usando recursão.} \\
    Escreva uma função recursiva para imprimir a inversão de uma string fornecida. \\
    \textbf{Exemplo:} \\
    Entrada: str = "hello" \\
    Saída: "olleh" \\
    \textbf{Caso de Teste:} \\
    Entrada: str = "recursao" \\
    Saída Esperada: "oasrucer"
    
    \item \textbf{Programa para comprimento de uma string usando recursão.} \\
    Crie uma função recursiva para encontrar o comprimento de uma string fornecida. \\
    \textbf{Exemplo:} \\
    Entrada: str = "hello" \\
    Saída: 5 \\
    \textbf{Caso de Teste:} \\
    Entrada: str = "recursao" \\
    Saída Esperada: 9
    
    \item \textbf{Soma dos dígitos de um número usando recursão.} \\
    Escreva uma função recursiva para calcular a soma dos dígitos de um número fornecido. \\
    \textbf{Exemplo:} \\
    Entrada: n = 1234 \\
    Saída: 10 \\
    \textbf{Caso de Teste:} \\
    Entrada: n = 9876 \\
    Saída Esperada: 30
    
    \item \textbf{Recursão de cauda para calcular a soma dos elementos do vetor.} \\
    Explique o que é recursão de cauda e escreva uma função recursiva de cauda para calcular a soma dos elementos do vetor. \\
    \textbf{Explicação:} A recursão de cauda ocorre quando a chamada recursiva é a última instrução a ser executada na função, permitindo otimização. \\
    \textbf{Exemplo:} \\
    Entrada: arr = [1, 2, 3, 4] \\
    Saída: 10 \\
    \textbf{Caso de Teste:} \\
    Entrada: arr = [5, 10, 15, 20] \\
    Saída Esperada: 50
    
    \item \textbf{Programa para imprimir os primeiros n números de Fibonacci.} \\
    Escreva uma função recursiva para imprimir os primeiros n números de Fibonacci. \\
    \textbf{Exemplo:} \\
    Entrada: n = 5 \\
    Saída: 0 1 1 2 3 \\
    \textbf{Caso de Teste:} \\
    Entrada: n = 7 \\
    Saída Esperada: 0 1 1 2 3 5 8
    
    \item \textbf{Programa para fatorial de um número.} \\
    Escreva uma função recursiva para calcular o fatorial de um número fornecido. \\
    \textbf{Exemplo:} \\
    Entrada: n = 5 \\
    Saída: 120 \\
    \textbf{Caso de Teste:} \\
    Entrada: n = 7 \\
    Saída Esperada: 5040
    
    \item \textbf{Programas Recursivos para encontrar os elementos Mínimos e Máximos de um vetor.} \\
    Escreva duas funções recursivas para encontrar os elementos mínimos e máximos de um vetor fornecido. \\
    \textbf{Exemplo:} \\
    Entrada: arr = [1, 4, 3, -5, 10] \\
    Saída: Mínimo = -5, Máximo = 10 \\
    \textbf{Caso de Teste:} \\
    Entrada: arr = [20, 15, 25, 5] \\
    Saída Esperada: Mínimo = 5, Máximo = 25
    
    \item \textbf{Função recursiva para verificar se uma string é um palíndromo.} \\
    Escreva uma função recursiva para verificar se uma string é um palíndromo. \\
    \textbf{Exemplo:} \\
    Entrada: str = "radar" \\
    Saída: Verdadeiro \\
    \textbf{Caso de Teste:} \\
    Entrada: str = "hello" \\
    Saída Esperada: Falso
    
    \item \textbf{Imprimir a Série de Fibonacci em ordem inversa usando Recursão.} \\
    Escreva uma função recursiva para imprimir a série de Fibonacci em ordem inversa. \\
    \textbf{Exemplo:} \\
    Entrada: n = 5 \\
    Saída: 3 2 1 1 0 \\
    \textbf{Caso de Teste:} \\
    Entrada: n = 7 \\
    Saída Esperada: 8 5 3 2 1 1 0
    
    \item \textbf{Troca de Moedas – Contar maneiras de fazer a soma.} \\
    Dada uma matriz inteira \texttt{coins[]} de tamanho N representando diferentes tipos de denominações e um inteiro \texttt{sum}, a tarefa é contar todas as combinações de moedas para formar um valor dado \texttt{sum}. Assuma um suprimento infinito de cada tipo de moeda.
    
    \textbf{Exemplos:} \\
    Entrada: sum = 4, coins[] = \{1, 2, 3\} \\
    Saída: 4 \\
    \textit{Explicação:} Há quatro soluções: \{1, 1, 1, 1\}, \{1, 1, 2\}, \{2, 2\} e \{1, 3\}.
    
    \item \textbf{Busca Binária usando Recursão.} \\
    Escreva uma função recursiva que implemente o algoritmo de busca binária para encontrar a posição de um elemento alvo em um vetor ordenado. Se o elemento alvo não estiver presente, a função deve retornar -1. 
    
    \textbf{Explicação do Algoritmo:} 
    - A busca binária funciona dividindo repetidamente o intervalo de busca pela metade. 
    - Se o valor da chave de busca for menor que o item no meio do intervalo, a busca continua na metade inferior. 
    - Se for maior, a busca continua na metade superior. 
    - Este processo continua até que o valor seja encontrado ou o intervalo se torne vazio.

    \textbf{Exemplos:} \\
    Entrada: arr = \{1, 2, 3, 4, 5, 6, 7, 8, 9, 10\}, alvo = 6 \\
    Saída: 5 \\
    \textit{Explicação:} O elemento 6 está no índice 5 (indexação baseada em 0).

\end{enumerate}

\end{document}
