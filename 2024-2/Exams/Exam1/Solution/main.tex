\documentclass[12pt]{article}
\usepackage{amsmath}
\usepackage{algorithm}
\usepackage{algpseudocode}
\usepackage{graphicx}
\usepackage{tikz}

\title{Prova de Projeto e Análise de Algoritmos}
\author{Disciplina: Projeto e Análise de Algoritmos}
\date{}

\begin{document}

\maketitle

\section*{Instruções}
Para cada uma das questões a seguir, escreva um algoritmo que resolve o problema apresentado, defina a operação básica, obtenha a expressão para o número de operações e defina a ordem de complexidade do algoritmo apresentado. Além disso, apresente exemplos de entrada e saída.

\section*{Questão 1}
\textbf{Problema:} Escreva um algoritmo que imprima a submatriz que possui a interseção das colunas ímpares com as linhas pares de uma matriz.

\textbf{Exemplo de Entrada:}
\[
\begin{bmatrix}
1 & 2 & 3 & 4 \\
5 & 6 & 7 & 8 \\
9 & 10 & 11 & 12 \\
13 & 14 & 15 & 16
\end{bmatrix}
\]

\textbf{Exemplo de Saída:}
\[
\begin{bmatrix}
2 & 4 \\
10 & 12
\end{bmatrix}
\]

\textbf{Algoritmo:}
\begin{algorithm}
\caption{Imprime colunas ímpares e linhas pares}
\begin{algorithmic}[1]
\Procedure{ImprimeColLinPar}{$M$}
    \For{cada linha $i$ de $M$ tal que $i$ é par}
        \For{cada coluna $j$ de $M$ tal que $j$ é ímpar}
            \State Imprimir $M[i][j]$
        \EndFor
    \EndFor
\EndProcedure
\end{algorithmic}
\end{algorithm}

\textbf{Operação básica:} Comparação de índices.

\textbf{Complexidade:} A complexidade do algoritmo é $O(m \times n)$, onde $m$ é o número de linhas e $n$ é o número de colunas.

\newpage

\section*{Questão 2}
\textbf{Problema:} Escreva um algoritmo que realiza a multiplicação de duas matrizes \(A\) e \(B\).

\textbf{Multiplicação de Matrizes:} Dados \(A\) de dimensão \(m \times n\) e \(B\) de dimensão \(n \times p\), o produto \(C = A \times B\) terá dimensão \(m \times p\). O elemento \(C[i][j]\) é calculado como:

\[
C[i][j] = \sum_{k=1}^{n} A[i][k] \times B[k][j]
\]

\textbf{Exemplo de Entrada:}
\[
A = \begin{bmatrix}
1 & 2 \\
3 & 4
\end{bmatrix},
B = \begin{bmatrix}
5 & 6 \\
7 & 8
\end{bmatrix}
\]

\textbf{Exemplo de Saída:}
\[
C = \begin{bmatrix}
19 & 22 \\
43 & 50
\end{bmatrix}
\]

\textbf{Algoritmo:}
\begin{algorithm}
\caption{Multiplicação de Matrizes}
\begin{algorithmic}[1]
\Procedure{MultiplicaMatrizes}{$A, B$}
    \For{cada linha $i$ de $A$}
        \For{cada coluna $j$ de $B$}
            \State $C[i][j] \gets 0$
            \For{cada elemento $k$ da linha $i$ de $A$}
                \State $C[i][j] \gets C[i][j] + A[i][k] \times B[k][j]$
            \EndFor
        \EndFor
    \EndFor
\EndProcedure
\end{algorithmic}
\end{algorithm}

\textbf{Operação básica:} Multiplicação de elementos da matriz.

\textbf{Complexidade:} A complexidade do algoritmo é $O(m \times n \times p)$, onde $m$, $n$ e $p$ são as dimensões das matrizes.

\newpage

\section*{Questão 3}
\textbf{Problema:} Dado um array \texttt{arr} de tamanho \(n-1\) que contém inteiros distintos no intervalo de 1 a \(n\), encontre o elemento faltante.

\textbf{Exemplo de Entrada:}
\[
arr = [1, 2, 4, 6, 3, 7, 8]
\]

\textbf{Exemplo de Saída:}
\[
5
\]

\textbf{Algoritmo:}
\begin{algorithm}
\caption{Encontra elemento faltante}
\begin{algorithmic}[1]
\Procedure{EncontraFaltante}{$arr, n$}
    \State $S \gets \frac{n(n+1)}{2}$ \Comment{Soma dos primeiros $n$ inteiros}
    \State $S_{arr} \gets \sum_{i=1}^{n-1} arr[i]$
    \State \Return $S - S_{arr}$
\EndProcedure
\end{algorithmic}
\end{algorithm}

\textbf{Operação básica:} Soma de elementos.

\textbf{Complexidade:} A complexidade do algoritmo é $O(n)$.

\newpage

\section*{Questão 4}
\textbf{Problema:} Dado um array \texttt{arr} de \(n\) inteiros, encontre todos os líderes no array. Um elemento é considerado um líder se ele for maior ou igual a todos os elementos à sua direita.

\textbf{Exemplo de Entrada:}
\[
arr = [16, 17, 4, 3, 5, 2]
\]

\textbf{Exemplo de Saída:}
\[
[17, 5, 2]
\]

\textbf{Algoritmo:}
\begin{algorithm}
\caption{Encontra Líderes}
\begin{algorithmic}[1]
\Procedure{EncontraLideres}{$arr$}
    \State $n \gets \text{tamanho de } arr$
    \State $max \gets arr[n-1]$
    \State Imprimir $max$
    \For{$i \gets n-2 \text{ até } 0$}
        \If{$arr[i] \geq max$}
            \State Imprimir $arr[i]$
            \State $max \gets arr[i]$
        \EndIf
    \EndFor
\EndProcedure
\end{algorithmic}
\end{algorithm}

\textbf{Operação básica:} Comparação entre elementos.

\textbf{Complexidade:} A complexidade do algoritmo é $O(n)$.


\end{document}
