\documentclass[12pt]{article}
\usepackage{amsmath}
\usepackage{algorithm}
\usepackage{algpseudocode}
\usepackage{graphicx}
\usepackage{tikz}
\usepackage{enumitem}
\usepackage[margin=1in]{geometry}

\title{\vspace{-1.5cm} Projeto e Análise de Algoritmos - Prova 2}
\author{}
\date{}

\begin{document}

\maketitle

\vspace{-2cm} 
\section*{Instruções}
Para cada uma das questões a seguir:
\begin{enumerate}[noitemsep] 
    \item Escreva um algoritmo que resolve o problema apresentado.
    \item Defina a operação básica.
    \item Obtenha a expressão para o número de operações.
    \item Defina a ordem de complexidade do algoritmo apresentado.  
\end{enumerate}

\section*{Questões}

\begin{enumerate}
    \item Dada uma matriz quadrada, escreve um algoritmo que verifique se as somas dos valores das colunas são iguais as somas da linhas. Isto é, o algoritmo deve retornar True se soma dos valores da coluna $i$ é igual a soma dos valores da linha $i$ para todo $i$ em $[0,1,2,...n-1]$ e False, caso contrário.
    
%    \textbf{Exemplos}
%
%    \textit{Entrada:}
%    \[
%    A = \begin{bmatrix}
%    1 & 3 \\
%    3 & 4
%    \end{bmatrix}
%    \]
%
%    \textit{Saída:}
%    True
%
%
%    \textit{Entrada:}
%    \[
%    A = \begin{bmatrix}
%    1 & 7 \\
%    3 & 4
%    \end{bmatrix}
%    \]
%
%    \textit{Saída:}
%    False


    \item Escreva uma função recursiva para calcular a soma dos primeiros n números naturais.
    
    \item Apresente um algoritmo baseado em busca exaustiva para o problema da Mochila binário (Binary Knapsack Problem). Você de deve apresentar e analisar tanto o algoritmo de busca quanto o método que calcula o custo de uma solução.
    
    \item Para da uma das operações à seguir aponte a ordem de complexidade $\Theta(\cdot)$:
    \begin{enumerate}
        \item Adicionar elemento no início de um array.
        \item Remover elemento no início de um array.
        \item Adicionar elemento no final de um array.
        \item Remover elemento no final de um array.
        \item Buscar elemento num array.
        \item Adicionar elemento numa pilha (implementada num array)
        \item Remover elemento de uma pilha (implementada num array)
        \item Adicionar elemento numa fila (implementada num array)
        \item Remover elemento de uma fila (implementada num array)
    \end{enumerate}

\end{enumerate}






\end{document}
