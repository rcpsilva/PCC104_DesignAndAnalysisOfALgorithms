\documentclass{article}
\usepackage[utf8]{inputenc}
\usepackage[margin=1.2in]{geometry}
\usepackage{hyperref}

\usepackage{listings}
\usepackage{xcolor}

\definecolor{codegreen}{rgb}{0,0.6,0}
\definecolor{codegray}{rgb}{0.5,0.5,0.5}
\definecolor{codepurple}{rgb}{0.58,0,0.82}
\definecolor{backcolour}{rgb}{0.95,0.95,0.92}

\lstdefinestyle{mystyle}{
    backgroundcolor=\color{backcolour},   
    commentstyle=\color{codegreen},
    keywordstyle=\color{magenta},
    numberstyle=\tiny\color{codegray},
    stringstyle=\color{codepurple},
    basicstyle=\ttfamily\footnotesize,
    breakatwhitespace=false,         
    breaklines=true,                 
    captionpos=b,                    
    keepspaces=true,                 
    numbers=left,                    
    numbersep=5pt,                  
    showspaces=false,                
    showstringspaces=false,
    showtabs=false,                  
    tabsize=2
}

\lstset{style=mystyle}


\usepackage{tikz}
\usetikzlibrary{positioning}

\usepackage{natbib}
\usepackage{graphicx}
\usepackage{amsmath}

\title{\vspace{-2 cm}Universidade Federal de Ouro Preto \\ PCC104 - Projeto e Análise de Algoritmos \\ Somatórios, PA e PG}
\author{Prof. Rodrigo Silva}
%\date{}


\begin{document}

\maketitle

\section*{Leitura Recomendada}

\begin{itemize}
    \item Important summation formulas (Appendix A) - \textit{Introduction to the Design and Analysis of Algorithms (3rd Edition)} - Anany Levitin 
    \item Sum manipulation rules (Appendix A) - \textit{Introduction to the Design and Analysis of Algorithms (3rd Edition)} - Anany Levitin 
    \item PA e PG: resumo, fórmulas e exercícios - \url{https://www.todamateria.com.br/pa-e-pg/}
\end{itemize}


\section{Atividades}

\subsection{Important summation formulas}

\begin{enumerate}

    \item $ S = \sum_{n=1}^{5} 3 $
    \item $ S = \sum_{k=0}^{15} (-2) $
    \item $ S = \sum_{i=3}^{8} 7 $
    \item $ S = \sum_{n=1}^{x} 3 $
    \item $ S = \sum_{k=0}^{n+2} (-2) $
    \item $ S = \sum_{i=3}^{k} 7 $
    \item $ S = \sum_{k=1}^{5} k$ 
    \item $ S = \sum_{k=0}^{n} k$
    \item $ S = \sum_{k=i}^{n} k$
    \item $ S = \sum_{i=1}^{4} i^2$
    \item $ S = \sum_{i=1}^{n} i^2 $
    
\end{enumerate}

\subsection{Sum manipulation rules}

\begin{enumerate}
    \item \(S = \sum_{n=1}^{6} (3n)\)
    \item \(S = \sum_{k=1}^{10} (-2k)\)
    \item \(S = \sum_{i=1}^{8} (5i)\)
    \item \(S = \sum_{n=1}^{k} (3n)\)
    \item \(S= \sum_{k=1}^{n} (-2k)\)
    \item \(S = \sum_{i=1}^{x} (5i)\)
    \item \(S = \sum_{n=1}^{5} (2n + 3)\)
    \item \(S = \sum_{k=1}^{7} (4k - 1)\)
    \item \(S = \sum_{n=1}^{k} (2n + 3)\)
    \item \(S = \sum_{k=1}^{n} (4k - 1)\)
    \item \(S = \sum_{n=1}^{6} (3n - 2n)\)
    \item \(S = \sum_{k=1}^{8} (2k + k+1)\)
    \item \(S = \sum_{i=1}^{k} (4i - 4i-1)\)
    \item \(S = \sum_{n=1}^{7} (2n) + \sum_{n=1}^{7} (3n)\)
    \item \(S = \sum_{k=1}^{5} (k^2) + \sum_{k=1}^{5} (2k)\)
    \item \(S = \sum_{i=1}^{6} (4i) + \sum_{i=3}^{6} (i^2)\).
\end{enumerate}

\subsection{Progressões aritméticas}

\begin{enumerate}

    \item  Seja \(S_n\) a soma dos primeiros \(n\) termos de uma progressão aritmética. Se o primeiro termo é 3 e a razão é 6, escreva \(S_n\) em termos de \(n\).
    
    \item  Dada a progressão aritmética \(8, 14, 20, \ldots\), determine o valor do 50º termo.
    
    \item  Seja \(a_1\) o primeiro termo de uma progressão aritmética e \(a_n\) o \(n\)-ésimo termo. Se a soma dos primeiros 12 termos é 234 e \(a_{12} = 31\), encontre \(a_1\).
    
    \item  A soma dos primeiros \(n\) termos de uma progressão aritmética é \(S_n = 3n^2 + 2n\). Determine a expressão para o \(n\)-ésimo termo \(a_n\) em termos de \(n\).
    
    \item  Em uma progressão aritmética, a soma dos primeiros 25 termos é igual a 500, e a soma dos primeiros 40 termos é igual a 880. Encontre o valor do primeiro termo e da razão da progressão.
    
\end{enumerate}

\subsection{Progressões geométricas}

\begin{enumerate}
    \item Seja \(S_n\) a soma dos primeiros \(n\) termos de uma progressão geométrica. Se o primeiro termo é 2 e a razão é \(3/2\), escreva \(S_n\) em termos de \(n\).
    
    \item  Dada a progressão geométrica \(5, 10, 20, \ldots\), determine o valor do 7º termo.
    
    \item  Seja \(a_1\) o primeiro termo de uma progressão geométrica e \(a_n\) o \(n\)-ésimo termo. Se a soma dos primeiros 8 termos é 546 e \(a_{8} = 32\), encontre \(a_1\).
    
    \item  A soma dos primeiros \(n\) termos de uma progressão geométrica é \(S_n = 80(2^n - 1)\). Determine a expressão para o \(n\)-ésimo termo \(a_n\) em termos de \(n\).
    
    \item  Em uma progressão geométrica, a soma dos primeiros 10 termos é igual a 511 e a soma dos primeiros 5 termos é igual a 455. Encontre o valor do primeiro termo e da razão da progressão.
    
\end{enumerate}


%\bibliographystyle{plain}
%\bibliography{references}
\end{document}
