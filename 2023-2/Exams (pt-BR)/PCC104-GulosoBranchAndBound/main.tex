\documentclass{article}
\usepackage[utf8]{inputenc}
\usepackage[margin=1.2in]{geometry}
\usepackage{hyperref}

\usepackage{listings}
\usepackage{xcolor}

\definecolor{codegreen}{rgb}{0,0.6,0}
\definecolor{codegray}{rgb}{0.5,0.5,0.5}
\definecolor{codepurple}{rgb}{0.58,0,0.82}
\definecolor{backcolour}{rgb}{.95,.95,.95}

\lstdefinestyle{mystyle}{
    backgroundcolor=\color{backcolour},   
    commentstyle=\color{codegreen},
    %keywordstyle=\color{magenta},
    numberstyle=\tiny\color{codegray},
    %stringstyle=\color{codepurple},
    %basicstyle=\ttfamily\footnotesize,
    breakatwhitespace=false,         
    breaklines=true,                 
    captionpos=b,                    
    keepspaces=true,                 
    numbers=left,                    
    numbersep=5pt,                  
    showspaces=false,                
    showstringspaces=false,
    showtabs=false,                  
    tabsize=2
}

\lstset{style=mystyle}


\usepackage{tikz}
\usetikzlibrary{positioning}

\usepackage{natbib}
\usepackage{graphicx}
\usepackage{amsmath}

\title{\vspace{-2 cm}Universidade Federal de Ouro Preto \\ PCC104 - Projeto e Análise de Algoritmos \\ Prova 5}
\author{Prof. Rodrigo Silva}
%\date{}


\begin{document}

\maketitle

\section*{Questões}

\begin{enumerate}

    \item Resolva as seguintes relações de recorrência
    
    \begin{enumerate}
        \item $T(n) = T(n-1) + n$, $T(1)=0$
        \item $T(n) = 2T(n/2) + 1$, $T(1)=1$
    \end{enumerate}

    \item Escreva um pseudo-código genérico do algoritmo de branch-and-bound.
    
    \item Que tipo de problema resolvemos com algoritmos gulosos? Descreva, em linguagem natural, como um algoritmo guloso chega à uma solução. 

    %\item Escreva um algoritmo recursivo que tenha custo definido por $T(n) = nT(n-1) + 1$, $T(1)=1$. 
   

    \item Escreva um algoritmo que retorne a diagonal principal de uma matriz e tenha complexidade $\Theta(n)$.

    \item Para cada um dos algoritmos abaixo, indique:
    
    \begin{enumerate}
        \item Qual a operação básica deste algoritmo?
        \item Quantas vezes esta operação básica é executada?
        \item Qual a classe deste algoritmo em relaçãoo à eficiência?
    \end{enumerate}


    \begin{figure}[!ht]
        \lstinputlisting[language=Python]{it2.py}
        \caption{Algoritmo 1}
    \end{figure}

    \begin{figure}[!ht]
        \lstinputlisting[language=Python]{rec2.py}
        \caption{Algoritmo 6}
    \end{figure}    

    \begin{figure}[!ht]
        \lstinputlisting[language=Python]{it4.py}
        \caption{Algoritmo 2}
    \end{figure}

      
    
\end{enumerate}



%\bibliographystyle{plain}
%\bibliography{references}
\end{document}
