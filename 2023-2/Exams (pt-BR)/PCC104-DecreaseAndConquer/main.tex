\documentclass{article}
\usepackage[utf8]{inputenc}
\usepackage[margin=1.2in]{geometry}
\usepackage{hyperref}

\usepackage{listings}
\usepackage{xcolor}

\definecolor{codegreen}{rgb}{0,0.6,0}
\definecolor{codegray}{rgb}{0.5,0.5,0.5}
\definecolor{codepurple}{rgb}{0.58,0,0.82}
\definecolor{backcolour}{rgb}{.95,.95,.95}

\lstdefinestyle{mystyle}{
    backgroundcolor=\color{backcolour},   
    commentstyle=\color{codegreen},
    %keywordstyle=\color{magenta},
    numberstyle=\tiny\color{codegray},
    %stringstyle=\color{codepurple},
    %basicstyle=\ttfamily\footnotesize,
    breakatwhitespace=false,         
    breaklines=true,                 
    captionpos=b,                    
    keepspaces=true,                 
    numbers=left,                    
    numbersep=5pt,                  
    showspaces=false,                
    showstringspaces=false,
    showtabs=false,                  
    tabsize=2
}

\lstset{style=mystyle}


\usepackage{tikz}
\usetikzlibrary{positioning}

\usepackage{natbib}
\usepackage{graphicx}
\usepackage{amsmath}

\title{\vspace{-2 cm}Universidade Federal de Ouro Preto \\ PCC104 - Projeto e Análise de Algoritmos \\ Prova 1}
\author{Prof. Rodrigo Silva}
%\date{}


\begin{document}

\maketitle

\section*{Orientações}

\begin{itemize}
    \item É obrigatória a entrega do código fonte dos algoritmos implementados. Provas sem os códigos fonte não serão corrigidas e terão nota 0.
    \item O código não deve conter nenhuma informação/comentário que auxilie na análise de complexidade do mesmo. 
    \item A avaliação do código apresentado entra na avaliação das questões relacionadas. 
    \item Simplificações feitas na análise de custo dos algoritmos devem ser indicadas e justificadas.
\end{itemize}

\section*{Questões}

\begin{enumerate}
    
    \item Considere a sua implementação do algoritmo de Busca Binária

    \begin{enumerate}
        \item (1.5 pt) Apresente a expressão matemática que define o custo em termos do número de comparações. Explique o que representa e de onde saiu cada um dos termos da sua expressão. 
        \item (1 pt) Resolva a expressão para encontrar o custo do algoritmo.
        \item (1 pt) Determine a classe de complexidade em notação $O$ ou $\Theta$ utilizando algum método formal. 
    \end{enumerate}
    
    \item Considere a sua implementação do algoritmo que resolve o problema da moeda falsa.
    \begin{enumerate}
        \item (1.5 pt) Apresente a expressão matemática que define o custo em termos do número de operações de adição. Explique o que representa e de onde saiu cada um dos termos da sua expressão. 
        \item (1 pt) Resolva a expressão para encontrar o custo do algoritmo.
        \item (1 pt)Determine a classe de complexidade em notação $O$ ou $\Theta$ utilizando algum método formal. 
    \end{enumerate}

    \item Considere a implementação de uma árvore de busca binária abaixo.
    
    \lstinputlisting[language=Python]{BST.py}

    \begin{enumerate}
        \item (1 pt) Apresente a expressão matemática que define o custo do algoritmo de busca (\texttt{search}) no \textbf{pior caso} em termos do número comparações. 
        \item (1 pt) Apresente a expressão matemática que define o custo do algoritmo de busca (\texttt{search}) no \textbf{melhor caso} termos do número comparações.
        \item (1 pt) O melhor caso do algoritmo de busca em uma árvore de busca binária é $\Theta(log_2n)$. Discuta vantagens e desvatagens desta estrutura de dados em relação a busca em um vetor utilizando busca binária ou não.  
    \end{enumerate}

\end{enumerate}



%\bibliographystyle{plain}
%\bibliography{references}
\end{document}
