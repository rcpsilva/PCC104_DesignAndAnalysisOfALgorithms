\documentclass{article}
\usepackage[utf8]{inputenc}
\usepackage[margin=1.2in]{geometry}
\usepackage{hyperref}

\usepackage{listings}
\usepackage{xcolor}

\definecolor{codegreen}{rgb}{0,0.6,0}
\definecolor{codegray}{rgb}{0.5,0.5,0.5}
\definecolor{codepurple}{rgb}{0.58,0,0.82}
\definecolor{backcolour}{rgb}{.95,.95,.95}

\lstdefinestyle{mystyle}{
    backgroundcolor=\color{backcolour},   
    commentstyle=\color{codegreen},
    %keywordstyle=\color{magenta},
    numberstyle=\tiny\color{codegray},
    %stringstyle=\color{codepurple},
    basicstyle=\ttfamily\footnotesize,
    breakatwhitespace=false,         
    breaklines=true,                 
    captionpos=b,                    
    keepspaces=true,                 
    numbers=left,                    
    numbersep=5pt,                  
    showspaces=false,                
    showstringspaces=false,
    showtabs=false,                  
    tabsize=2
}

\lstset{style=mystyle}


\usepackage{tikz}
\usetikzlibrary{positioning}

\usepackage{natbib}
\usepackage{graphicx}
\usepackage{amsmath}

\title{\vspace{-2 cm}Universidade Federal de Ouro Preto \\ PCC104 - Projeto e Análise de Algoritmos \\ Prova 6}
\author{Prof. Rodrigo Silva}
%\date{}


\begin{document}

\maketitle

\section*{Questões}

\begin{enumerate}

    \item Apresente a análise de complexidade completa dos algoritmos abaixo. 
    
    \begin{enumerate}
        \item Algoritmo 1
        \begin{figure}[!ht]
            \lstinputlisting[language=Python]{algorithm1.py}
        \end{figure}

        \item Algoritmo 2
        \begin{figure}[!ht]
            \lstinputlisting[language=Python]{algorithm2.py}
        \end{figure}

        \item Algoritmo 3
        \begin{figure}[!ht]
            \lstinputlisting[language=Python]{algorithm3.py}
        \end{figure}

        \item Algoritmo 4
        \begin{figure}[!ht]
            \lstinputlisting[language=Python]{algorithm4.py}
        \end{figure}
    \end{enumerate}

    \item Faça a análise do resultados obtidos por você com as implementações do Branch and Bound e da Busca Exaustiva para o problema do caixeiro viajante. Demonstre conhecimento sobre os algoritmos implementados e sobre a complexidade de cada um.
    
    \item Dado um problema NP-completo, $P_{NPC}$, e um problema, ${P_{?}}$, que se deseja provar NP-completo responda:
    \begin{enumerate}
        \item Quais são os dois passos para se provar que $P_{?}$? 
        \item Como a prova de cada passo pode ser feita?
        \item Por quê os dois passos são necessários? Seria possível determinar a NP-Completude de $P_{?}$ só com o passo 1 ou só com o passo 2?
        \item Imagine que $P_{NPC}$ foi resolvido por algum algoritmo em tempo polinomial. Qual seria o custo de se resolver $P2_{NPC}$ que sabemos ser NP-completo? 
    \end{enumerate} 
    


    
\end{enumerate}



%\bibliographystyle{plain}
%\bibliography{references}
\end{document}
