\documentclass{article}
\usepackage[utf8]{inputenc}
\usepackage[margin=1.2in]{geometry}
\usepackage{hyperref}

\usepackage{listings}
\usepackage{xcolor}

\definecolor{codegreen}{rgb}{0,0.6,0}
\definecolor{codegray}{rgb}{0.5,0.5,0.5}
\definecolor{codepurple}{rgb}{0.58,0,0.82}
\definecolor{backcolour}{rgb}{0.95,0.95,0.92}

\lstdefinestyle{mystyle}{
    backgroundcolor=\color{backcolour},   
    commentstyle=\color{codegreen},
    keywordstyle=\color{magenta},
    numberstyle=\tiny\color{codegray},
    stringstyle=\color{codepurple},
    basicstyle=\ttfamily\footnotesize,
    breakatwhitespace=false,         
    breaklines=true,                 
    captionpos=b,                    
    keepspaces=true,                 
    numbers=left,                    
    numbersep=5pt,                  
    showspaces=false,                
    showstringspaces=false,
    showtabs=false,                  
    tabsize=2
}

\lstset{style=mystyle}


\usepackage{tikz}
\usetikzlibrary{positioning}

\usepackage{natbib}
\usepackage{graphicx}
\usepackage{amsmath}
\usepackage[utf8]{inputenc}
\usepackage{listings}


\title{\vspace{-2 cm}Universidade Federal de Ouro Preto \\ PCC104 - Projeto e Análise de Algoritmos \\ Prova - Algoritmos Gulosos}
\author{Prof. Rodrigo Silva}
%\date{}


\begin{document}

\maketitle

\section*{Orientações}

\begin{itemize}
    \item É obrigatória a entrega do código fonte dos algoritmos gulosos. Provas sem os códigos fonte não serão corrigidas e terão nota 0.
    \item A avaliação do código apresentado entra na avaliação das questões relacionadas. 
\end{itemize}


\section*{Questões}

\begin{enumerate}
    

\item Análise de Algoritmo Iterativo Simples

Considerando o algoritmo de ordenação abaixo, que implementa o método de "Bubble Sort":

\begin{lstlisting}[language=Python]
def bubble_sort(lista):
    for i in range(len(lista)):
        for j in range(0, len(lista) - i - 1):
            if lista[j] > lista[j + 1]:
                lista[j], lista[j + 1] = lista[j + 1], lista[j]
\end{lstlisting}

\noindent
b) Qual é a complexidade de tempo no pior caso deste algoritmo? \\
c) Qual é a complexidade de espaço deste algoritmo?

\item Análise de Algoritmo Recursivo Simples

Veja o seguinte algoritmo de busca binária:

\begin{lstlisting}[language=Python]
def binary_search(array, low, high, target):
    if high >= low:
        mid = (high + low) // 2
        if array[mid] == target:
            return mid
        elif array[mid] > target:
            return binary_search(array, low, mid - 1, target)
        else:
            return binary_search(array, mid + 1, high, target)
    else:
        return -1
\end{lstlisting}

\noindent
a) Explique como este algoritmo funciona. \\
b) Determine e justifique a complexidade de tempo no pior caso deste algoritmo. \\
c) Determine e justifique a complexidade de tempo no melhor caso deste algoritmo. \\

\item Análise e Perguntas Teóricas Sobre o Branch and Bound

\noindent
a) Explique o conceito de branch and bound e sua aplicação na resolução de problemas de otimização. \\
b) Compare o branch and bound com o método de força bruta. Em quais cenários cada um seria preferível e por quê? \\
c) Como a estratégia de branch and bound pode impactar o custo computacional da resolução de um problema? \\
d) Apresente a análise de custo da sua implementação do branch and bound.

\item Perguntas Teóricas Sobre Classes de Problemas (P, NP, NP-completo)

\noindent
a) Defina as classes de problemas P, NP e NP-Completo. \\
c) É possível que P = NP? Explique sua resposta, discutindo as implicações se P fosse de fato igual a NP. 

\end{enumerate}


%\bibliographystyle{plain}
%\bibliography{references}
\end{document}
