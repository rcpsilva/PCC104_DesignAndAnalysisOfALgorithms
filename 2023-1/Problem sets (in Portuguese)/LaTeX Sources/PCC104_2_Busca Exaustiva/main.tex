\documentclass{article}
\usepackage[utf8]{inputenc}
\usepackage[margin=1.2in]{geometry}
\usepackage{hyperref}

\usepackage{listings}
\usepackage{xcolor}

\definecolor{codegreen}{rgb}{0,0.6,0}
\definecolor{codegray}{rgb}{0.5,0.5,0.5}
\definecolor{codepurple}{rgb}{0.58,0,0.82}
\definecolor{backcolour}{rgb}{0.95,0.95,0.92}

\lstdefinestyle{mystyle}{
    backgroundcolor=\color{backcolour},   
    commentstyle=\color{codegreen},
    keywordstyle=\color{magenta},
    numberstyle=\tiny\color{codegray},
    stringstyle=\color{codepurple},
    basicstyle=\ttfamily\footnotesize,
    breakatwhitespace=false,         
    breaklines=true,                 
    captionpos=b,                    
    keepspaces=true,                 
    numbers=left,                    
    numbersep=5pt,                  
    showspaces=false,                
    showstringspaces=false,
    showtabs=false,                  
    tabsize=2
}

\lstset{style=mystyle}


\usepackage{tikz}
\usetikzlibrary{positioning}

\usepackage{natbib}
\usepackage{graphicx}
\usepackage{amsmath}

\title{\vspace{-2 cm}Universidade Federal de Ouro Preto \\ PCC104 - Projeto e Análise de Algoritmos \\ Força Bruta e Busca Exaustiva}
\author{Prof. Rodrigo Silva}
%\date{}


\begin{document}

\maketitle

\section*{Instruções}



% \begin{itemize}
%     \item Implementar as atividades práticas em C++ é altamente recomendado.
%     \item Utilize ao máximo os algoritmos e estruturas de dados da biblioteca STL. \url{https://www.geeksforgeeks.org/the-c-standard-template-library-stl/}. 
%     \item Evite ao máximo a utilização de ponteiros, mas se precisar, utilizar ponteiros inteligentes \url{https://alandefreitas.github.io/moderncpp/basic-syntax/pointers/smart-pointers/}. 
%     \item Quando precisar de uma estrutura de dados linear sempre avalie primeiro a utilização da classe \texttt{vector} (\url{https://en.cppreference.com/w/cpp/container/vector})
% \end{itemize}

\section{Leitura Recomendada}

\begin{itemize}
    \item Capítulo 3 - \textit{Introduction to the Design and Analysis of Algorithms (3rd Edition)} - Anany Levitin 
    \item Livro - \textit{Problem Solving with Algorithms and Data Structures using C++} (disponível em: \url{https://runestone.academy/runestone/books/published/cppds/index.html#})
    \item Arrays \url{https://superstudy.guide/algorithms-data-structures/data-structures/arrays-strings}
    \item Pilhas e Filas \url{https://superstudy.guide/algorithms-data-structures/data-structures/stacks-queues}
    \item Livro - \textit{Introdução à programação} - Alan de Freitas (disponível em \url{http://www.decom.ufop.br/alan/bcc702/livrocpp.pdf})
\end{itemize}


\section{Atividades Práticas}

\begin{enumerate}
    \item Implementar o algoritmo \textit{Selection Sort}
    \item Implementar o algoritmo \textit{SequentialSearch2} (Ver Seção 3.2  \textit{Introduction to the Design and Analysis of Algorithms (3rd Edition)} - Anany Levitin).
    \item Implementar o algoritmo de busca em largura para grafos.
    \item Implementar o algoritmo de busca em profundidade para grafos.
    
    %\item Implemente uma solução baseada em busca exaustiva para o problema do Caixeiro Viajante (Traveling Salesman Problem). 
    %\item Implemente uma solução baseada em busca exaustiva para o problema da Mochila binário (Binary Knapsack Problem).
\end{enumerate}

Para cada implementação, apresentar a análise de complexidade de tempo do algoritmo.

%\footnotetext{Livro - \textit{Introduction to the Design and Analysis of Algorithms (3rd Edition)}}

%\bibliographystyle{plain}
%\bibliography{references}
\end{document}

