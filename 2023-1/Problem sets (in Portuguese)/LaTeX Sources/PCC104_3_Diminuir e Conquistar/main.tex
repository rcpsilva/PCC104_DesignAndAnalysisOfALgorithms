\documentclass{article}
\usepackage[utf8]{inputenc}
\usepackage[margin=1.2in]{geometry}
\usepackage{hyperref}

\usepackage{listings}
\usepackage{xcolor}

\definecolor{codegreen}{rgb}{0,0.6,0}
\definecolor{codegray}{rgb}{0.5,0.5,0.5}
\definecolor{codepurple}{rgb}{0.58,0,0.82}
\definecolor{backcolour}{rgb}{0.95,0.95,0.92}

\lstdefinestyle{mystyle}{
    backgroundcolor=\color{backcolour},   
    commentstyle=\color{codegreen},
    keywordstyle=\color{magenta},
    numberstyle=\tiny\color{codegray},
    stringstyle=\color{codepurple},
    basicstyle=\ttfamily\footnotesize,
    breakatwhitespace=false,         
    breaklines=true,                 
    captionpos=b,                    
    keepspaces=true,                 
    numbers=left,                    
    numbersep=5pt,                  
    showspaces=false,                
    showstringspaces=false,
    showtabs=false,                  
    tabsize=2
}

\lstset{style=mystyle}


\usepackage{tikz}
\usetikzlibrary{positioning}

\usepackage{natbib}
\usepackage{graphicx}
\usepackage{amsmath}

\title{\vspace{-2 cm}Universidade Federal de Ouro Preto \\ PCC104 - Projeto e Análise de Algoritmos \\ Diminuir e Conquistar}
\author{Prof. Rodrigo Silva}
%\date{}


\begin{document}

\maketitle

\section*{Instruções}

\begin{itemize}
    \item Evite ao máximo a utilização de ponteiros, mas se precisar, utilizar ponteiros inteligentes \url{https://alandefreitas.github.io/moderncpp/basic-syntax/pointers/smart-pointers/}. 
    \item Quando precisar de uma estrutura de dados linear sempre avalie primeiro a utilização da classe \texttt{vector} (\url{https://en.cppreference.com/w/cpp/container/vector})
\end{itemize}

\section{Leitura Recomendada}

\begin{itemize}
    \item Capítulo 4 - \textit{Introduction to the Design and Analysis of Algorithms (3rd Edition)} - Anany Levitin 
    \item Livro - \textit{Introdução à programação} - Alan de Freitas (disponível em \url{http://www.decom.ufop.br/alan/bcc702/livrocpp.pdf})
    \item Livro - \textit{Problem Solving with Algorithms and Data Structures using C++} (disponível em: \url{https://runestone.academy/runestone/books/published/cppds/index.html#})
\end{itemize}

\section{Vídeos Recomendados}
\begin{itemize}
    \item Confira a playlist de C++ do Prof. Alan de Freitas (UFOP) - \url{https://www.youtube.com/watch?v=jes0Z6i-3DA&list=PLIUc9-A-aPpqrzY3YuWDUOyQLOBCb5lck}
\end{itemize}

\section{Questões teóricas}

\begin{enumerate}
    \item Apresente uma descrição da classe \texttt{list} (\url{https://en.cppreference.com/w/cpp/container/list}) apresentando o custo computacional de cada um de suas operações.
    \item Apresente uma descrição da classe \texttt{set} (\url{https://en.cppreference.com/w/cpp/container/set}) apresentando o custo computacional de cada um de suas operações.
\end{enumerate}

\section{Atividades Práticas}

\begin{enumerate}
    \item Implemente o algortimo \textit{Insertion Sort}.
    %\item Implemente o algoritmo de busca topológia.
    \item Implemente o algoritmo de Johnson-Trotter para gerar permutações.
    \item Implemente o método \textit{LexicographicPermute} apresentado no livro texto.
    \item Implemente um algoritmo que, dado um conjunto de elementos, gere todos seus subconjuntos.
    \item Implemente o algoritmo de busca binária.
    \item Implemente o método \textit{interpolation search}.
    \item Implemente um algoritmo para o problema da moeda falsa (\textit{fake coin problem}).
    \item Implemente um algoritmo para o cálculo da mediana que não envolva ordenar o conjunto de números.
    \item Implemente a estrutura de dados \textit{binary search tree} e os métodos buscar e inserir.
    % https://hub.packtpub.com/binary-search-tree-tutorial/
\end{enumerate}

\footnotetext{Livro - \textit{Introduction to the Design and Analysis of Algorithms (3rd Edition)}}

%\bibliographystyle{plain}
%\bibliography{references}
\end{document}

