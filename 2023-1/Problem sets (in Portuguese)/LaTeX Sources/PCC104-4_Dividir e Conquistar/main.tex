\documentclass{article}
\usepackage[utf8]{inputenc}
\usepackage[margin=1.2in]{geometry}
\usepackage{hyperref}

\usepackage{listings}
\usepackage{xcolor}

\definecolor{codegreen}{rgb}{0,0.6,0}
\definecolor{codegray}{rgb}{0.5,0.5,0.5}
\definecolor{codepurple}{rgb}{0.58,0,0.82}
\definecolor{backcolour}{rgb}{0.95,0.95,0.92}

\lstdefinestyle{mystyle}{
    backgroundcolor=\color{backcolour},   
    commentstyle=\color{codegreen},
    keywordstyle=\color{magenta},
    numberstyle=\tiny\color{codegray},
    stringstyle=\color{codepurple},
    basicstyle=\ttfamily\footnotesize,
    breakatwhitespace=false,         
    breaklines=true,                 
    captionpos=b,                    
    keepspaces=true,                 
    numbers=left,                    
    numbersep=5pt,                  
    showspaces=false,                
    showstringspaces=false,
    showtabs=false,                  
    tabsize=2
}

\lstset{style=mystyle}


\usepackage{tikz}
\usetikzlibrary{positioning}

\usepackage{natbib}
\usepackage{graphicx}
\usepackage{amsmath}

\title{\vspace{-2 cm}Universidade Federal de Ouro Preto \\ PCC104 - Projeto e Análise de Algoritmos \\ Dividir e Conquistar}
\author{Prof. Rodrigo Silva}
%\date{}


\begin{document}

\maketitle

\section{Leitura Recomendada}

\begin{itemize}
    \item Capítulo 4 - \textit{Introduction to the Design and Analysis of Algorithms (3rd Edition)} - Anany Levitin 
    \item Livro - \textit{Introdução à programação} - Alan de Freitas (disponível em \url{http://www.decom.ufop.br/alan/bcc702/livrocpp.pdf})
    \item Livro - \textit{Problem Solving with Algorithms and Data Structures using C++} (disponível em: \url{https://runestone.academy/runestone/books/published/cppds/index.html#})
\end{itemize}


\section{Atividades}

\begin{enumerate}
    \item Implemente um algoritmo de divisão e conquista para encontrar a posição do maior elemento de um arranjo de $n$ elementos. Responda também: % 1 - 5.1
    \begin{enumerate} 
        \item Qual será a saída do método quando se vários elementos do arranjo tiverem o maior valor?
        \item Defina e resolva a relação de recorrência para o método proposto.
        \item Como este algoritmo se compra com uma solução força bruta?
    \end{enumerate}
    \item Implemente o algoritmo MergeSort. O MergeSort é um algoritmo estável? Explique.
    \item Implemente o algoritmo QuickSort. O QuickSort é um algoritmo estável? Explique.
    \item Implemente um algoritmo recursivo que encontre o tamanho de uma árvore binária.
    \item Implemente os caminhamentos \textit{preorder}, \textit{postorder} e \textit{inorder} para árvores binárias.
     
\end{enumerate}

Para cada implementação, apresentar a análise de complexidade de tempo do algoritmo. Esta análise deverá conter:

\begin{itemize}
    \item Expressão matemática que define o custo do algoritmo (relação de recorrência para recursivos ou somatórios para iterativos) 
    \item Cálculo da função de custo (quando possível, utilizar o teorema mestre para verificar o cálculo).
    \item Indicação da classe de eficiência ($O$ ou $\Theta$). A indicação da classe, deve ser justificada. Você pode provar pela definição, pelo limite, teorema mestre, utilizar os resultados demonstrados em aula.
\end{itemize}


%\bibliographystyle{plain}
%\bibliography{references}
\end{document}
