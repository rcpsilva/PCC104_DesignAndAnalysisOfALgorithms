\documentclass{article}
\usepackage[utf8]{inputenc}
\usepackage[margin=1.2in]{geometry}
\usepackage{hyperref}

\usepackage{listings}
\usepackage{xcolor}

\definecolor{codegreen}{rgb}{0,0.6,0}
\definecolor{codegray}{rgb}{0.5,0.5,0.5}
\definecolor{codepurple}{rgb}{0.58,0,0.82}
\definecolor{backcolour}{rgb}{0.95,0.95,0.92}

\lstdefinestyle{mystyle}{
    backgroundcolor=\color{white},   
    commentstyle=\itshape\color{green!40!black},
    keywordstyle=\color{blue},
    numberstyle=\tiny\color{codegray},
    stringstyle=\color{codepurple},
    basicstyle=\ttfamily\footnotesize,
    breakatwhitespace=false,         
    breaklines=true,                 
    captionpos=b,                    
    keepspaces=true,                 
    numbers=left,                    
    numbersep=5pt,                  
    showspaces=false,                
    showstringspaces=false,
    showtabs=false,                  
    tabsize=2
}

\lstset{style=mystyle}


\usepackage{tikz}
\usetikzlibrary{positioning}

\usepackage{natbib}
\usepackage{graphicx}
\usepackage{amsmath}

\title{\vspace{-2 cm}Universidade Federal de Ouro Preto \\ Lecture Notes \\ Graph Representation}
\author{Prof. Rodrigo Silva}
%\date{}


\begin{document}

\maketitle

\section*{Source}

\begin{itemize}
    \item Arrays \url{https://superstudy.guide/algorithms-data-structures/data-structures/arrays-strings}
    \item Stacks and Queues \url{https://superstudy.guide/algorithms-data-structures/data-structures/stacks-queues}
\end{itemize}

\section{Arrays}

An array is a collection of elements of the same data type that are stored together in contiguous memory locations and can be accessed using an index or a subscript.

\begin{figure}[!ht]
    \centering
    \includegraphics*[width=0.2\textwidth]{images/array.png}
\end{figure}

\subsection{Array operations}

\begin{figure}[!ht]
    \centering
    \includegraphics*[width=0.8\textwidth]{images/array_operations.png}
\end{figure}

\section{Stacks (Pilha pt-BR)} 

A stack is an abstract data type that represents a collection of elements with a particular set of operations. It is based on the principle of Last-In-First-Out (LIFO), which means that the last element added to the stack is the first one to be removed.

\section{Queues (Fila pt-BR)}

A queue is an abstract data type that represents a collection of elements with a particular set of operations. It is based on the principle of First-In-First-Out (FIFO), which means that the first element added to the queue is the first one to be removed.



% \lstinputlisting[language=Python,caption=Example 1 - Graph as Matrix Python]{graph_matrix.py}


\end{document}

